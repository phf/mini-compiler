\documentclass[letterpaper,12pt,twocolumn]{article}

\usepackage{hyperref}
\usepackage{times}
\usepackage{ftnright}
\usepackage{verbatim}

\usepackage[USletter]{vmargin}
\setmargrb{1in}{1in}{1in}{1in}

\usepackage{svn}
\SVN$Rev: 538 $
\SVN$Date: 2005-04-22 22:12:18 -0700 (Fri, 22 Apr 2005) $

\usepackage{color}
\definecolor{darkblue}{rgb}{0.0,0.0,0.5}
\definecolor{darkgreen}{rgb}{0.0,0.3,0.0}
\definecolor{darkred}{rgb}{0.5,0.0,0.0}
\hypersetup{
colorlinks=true,
urlcolor=darkblue,% urls
citecolor=darkred,% citation of reference
linkcolor=darkgreen,% table of contents
}

\title{
\textbf{The Programming Language Mini}%
\thanks{This document is based in part on documentation originally
written by \href{http://www.fridi.de/}{Fridtjof Siebert}. Any errors
or omissions are of course Peter's fault. :-)}
}
\author{
\href{http://www.factorial.com/forward/phf/work/}{Peter H. Fr{\"o}hlich}\\
\href{mailto:phf@acm.org}{phf@acm.org}
}
\date{\SVNDate\\(Revision \SVNRev)}

\begin{document}
\maketitle

\section{The Language}

Mini is a very simple programming language.
%
Programs consist of the keyword \texttt{PROGRAM} followed
by (optional) variable declarations, followed by the keyword
\texttt{BEGIN}, followed by (optional) instructions, followed
by the keyword \texttt{END}:
%
\begin{verbatim}
  Program =
    "PROGRAM" [VarDeclaration]
    "BEGIN" {Statement} "END" .
\end{verbatim}
%
Variable declarations are introduced with the keyword
\texttt{VAR} and list the variables separated by commas:
%
\begin{verbatim}
  VarDeclaration =
    "VAR" Identifier
    {"," Identifier}.
\end{verbatim}

\section{The Compiler}


\section{The Backends}

  

\end{document}













               Mini
            ==========


Dieses Verzeichnis enth�lt den Mini-Compiler, der im Amiga Programmieren
Sonderheft (Markt&Technik) beschrieben wird.

Dieser Compiler d�rfte mit weniger als 8KB Quelltext und 15KB Programmcode
einer der kleinsten Compiler �berhaupt sein. Dennoch enth�lt er alle Teile,
die auch in einem 'gro�en' Compiler zu finden sind.

Die Sprache Mini:

  Mini ist eine sehr einfache Programmiersprache. Ein Mini-Programm
  besteht aus dem Schl�sselwort PROGRAM, einem optionalen
  Varialbendeklarationsteil, dem Schl�sselwort BEGIN, den Anweisungen
  und END:
  
    PROGRAM [VarDeclaration] BEGIN {Statement} END
  
  Der Variablendeklarationsteil wird durch VAR eingeleitet und listet
  die Namen der Variablen, die durch Kommas getrennt werden, auf:
  
    VAR Identifier { "," Identifier }
  
  An Anweisungen kennt Mini lediglich die Zuweisung eines Wertes an eine
  Variable mit '=':
  
    Identifier "=" Expression
  
  eine einfache Schleife, die solange ausgef�hrt wird, wie der Ausdruck
  hinter WHILE gr��er als null ist:
  
    WHILE Expression DO { Statement } END
  
  und eine Ausgabeanweisung f�r Zahlen
  
    PRINT Expression
  
  In einem Ausdruck k�nnen die Faktoren lediglich addiert oder subtrahiert
  werden:
  
    [ "+" | "-" ] Factor { [ "+" | "-" ] Factor }
  
  Faktoren sind entweder Variablenbezeichner oder Konstanten:
  
    Identifier | Constant
  
  Die EBNF-Syntax von Mini ist in der Datei Grammatik.ebnf gegeben.
  

Benutzung des Compilers:

  Der Compiler erwartet den Namen einer Mini-Quelltextdatei als Argument.
  Die Ausgabe des Compilers ist der erzeugte Assembler-Code. Er sollte in
  eine Datei umgeleitet werden. Bsp:

    Mini >Fibonacci.s Fibonacci.mini

  Nun mu� der Text noch assembliert werden (etwa mit a68k von Fish 521):

    a68k Fibonacci.s

  zuletzt wird mit

    OLink FROM Fibonacci.o TO Fibonacci

  oder mit

    BLink Fibonacci.o TO Fibonacci

  ein ausf�hrbares Programm erzeugt.


Beispielprogramme:

  Fibonacci.mini:

    Berechnet die ersten 46 Fibonacci-Zahlen

  Fak.mini:

    Berechnet die Fakult�ten 1! bis 12!

  GGT.mini:

    Berechnet den GGT zweier Zahlen (Konstanten im Programm)

  Prim.mini:

    Bestimmt die Primzahlen unter den Zahlen 1 bis 1000.


Weitere Mini-Programme:

  Wie man an Prim.mini sieht, k�nnen mit dieser einfachen Sprache auch recht
  Komplexe Dinge berechnet werden. Wer weitere Mini-Programme schreibt,
  die interessante Dinge berechnen, der schicke mit bitte seine Mini-
  Quelltexte.

  Theoretisch sind alle Berechnungen, die mit irgendeiner Programmiersprache
  formuliert werden k�nnen, auch mit Mini l�sbar. Schwierig ist es nur,
  auch in Mini effiziente L�sungen zu finden.


Viel Spa� mit dem Mini-Compiler!


---  Fridtjof.
                                                             Fridtjof Siebert
                                                                 Nobileweg 67
                                                            7000 Stuttgart 40
                                   siebert@minnie.informatik.uni-stuttgart.de
                                                    fridi@amokst.adsp.sub.org

