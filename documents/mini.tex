%%%%%%%%%%%%%%%%%%%%%%%%%%%%%%%%%%%%%%%%%%%%%%%%%%%%%%%%%%%%%%%%%%%%%%
%
% $Id: mini.tex 622 2005-05-06 00:30:26Z phf $
%
% Copyright (c) 2005 by Peter H. Froehlich <phf@acm.org>.
% All rights reserved.
%
%%%%%%%%%%%%%%%%%%%%%%%%%%%%%%%%%%%%%%%%%%%%%%%%%%%%%%%%%%%%%%%%%%%%%%

\documentclass[letterpaper,11pt,twocolumn]{article}

\usepackage{hyperref}
\usepackage{times}
\usepackage{ftnright}
\usepackage{verbatim}
\usepackage{graphicx}

\usepackage{float}
\floatstyle{ruled}
\restylefloat{figure}

%\usepackage[USletter]{vmargin}
%\setmargrb{1in}{1in}{1in}{1in}

\usepackage{svn}
\SVN$Rev: 622 $
\SVN$Date: 2005-05-05 17:30:26 -0700 (Thu, 05 May 2005) $

\usepackage{color}
\definecolor{darkblue}{rgb}{0.0,0.0,0.5}
\definecolor{darkgreen}{rgb}{0.0,0.3,0.0}
\definecolor{darkred}{rgb}{0.5,0.0,0.0}
\hypersetup{
colorlinks=true,
urlcolor=darkblue,% urls
citecolor=darkred,% citation of reference
linkcolor=darkgreen,% table of contents
}

\title{
\textbf{The Programming Language Mini}%
\thanks{This document is based in part on documentation originally
written by \href{http://www.fridi.de/}{Fridtjof Siebert}. Any errors
or omissions are of course Peter's fault. :-)}
}
\author{
\href{http://www.factorial.com/forward/phf/work/}{Peter H. Fr{\"o}hlich}\\
\href{mailto:phf@acm.org}{phf@acm.org}
}
\date{\SVNDate\\(Revision \SVNRev)}

%%%%%%%%%%%%%%%%%%%%%%%%%%%%%%%%%%%%%%%%%%%%%%%%%%%%%%%%%%%%%%%%%%%%%%

\begin{document}
\maketitle
\tableofcontents

%%%%%%%%%%%%%%%%%%%%%%%%%%%%%%%%%%%%%%%%%%%%%%%%%%%%%%%%%%%%%%%%%%%%%%

\section{The Language}

Mini is a very simple programming language.
%
In theory all computations that can be performed using any
other programming language can also be performed in Mini.
%
However, it can be rather difficult to find efficient solutions
in Mini.

%%%%%%%%%%

\subsection{Programs}

Programs consist of the keyword \texttt{PROGRAM} followed
by (optional) variable declarations, followed by the keyword
\texttt{BEGIN}, followed by (optional) instructions, followed
by the keyword \texttt{END}:
%
\begin{quote}
\begin{verbatim}
Program =
  "PROGRAM" [VarDeclaration]
  "BEGIN" {Instruction}
  "END".
\end{verbatim}
\end{quote}

%%%%%%%%%%

\subsection{Declarations}

Variable declarations are introduced with the keyword
\texttt{VAR} followed by the list of variables separated
by commas:
%
\begin{quote}
\begin{verbatim}
VarDeclaration =
  "VAR" Identifier
  {"," Identifier}.
\end{verbatim}
\end{quote}
%
Identifiers consist of one or more letters and
are case-sensitive:
%
\begin{quote}
\begin{verbatim}
Identifier =
  Letter {Letter}.
Letter =
  "a"|"b"|...|"z"|
  "A"|"B"|...|"Z".
\end{verbatim}
\end{quote}

%%%%%%%%%%

\subsection{Instructions}

There are only three possible instructions in Mini:
%
\begin{quote}
\begin{verbatim}
Instruction =
  Assignment | While | Print.
\end{verbatim}
\end{quote}
%
Assignments store the value of an expression in a variable:
%
\begin{quote}
\begin{verbatim}
Assignment =
  Identifier "=" Expression.
\end{verbatim}
\end{quote}
%
While loops are executed as long as the value of the
controlling expression is greater than zero:
%
\begin{quote}
\begin{verbatim}
While =
  "WHILE" Expression
  "DO" {Instruction} "END".
\end{verbatim}
\end{quote}
%
Print instructions write the value of an expression to
standard output:
%
\begin{quote}
\begin{verbatim}
Print =
  "PRINT" Expression.
\end{verbatim}
\end{quote}

%%%%%%%%%%

\subsection{Expressions}

Expressions allow addition and subtraction of factors,
nothing else:
%
\begin{quote}
\begin{verbatim}
Expression =
  ["+"|"-"] Factor
  {("+"|"-") Factor}.
\end{verbatim}
\end{quote}
%
Factors are either identifiers denoting variables or
constants:
%
\begin{quote}
\begin{verbatim}
Factor =
  Identifier | Constant.
\end{verbatim}
\end{quote}
%
Constants consist of one or more digits and denote
an integer value in base 10:
%
\begin{quote}
\begin{verbatim}
Constant =
  Digit {Digit}.
Digit =
  "0"|"1"|...|"9".
\end{verbatim}
\end{quote}

%%%%%%%%%%

\subsection{Example}

Here is a simple example program showing off all of
Mini's features (except for unary negation):
%
\begin{quote}
\verbatiminput{../examples/count.mini}
\end{quote}
%
We initialize the variable \texttt{x} to the value \texttt{10};
we then execute a loop as long as \texttt{x+1} is greater than
\texttt{0}
(that is, as long as \texttt{x} is greater than \texttt{-1});
inside the loop with print the current value of \texttt{x} and
then subtract \texttt{1} from \texttt{x}
before the next iteration.
%
The resulting program ``counts down'' from \texttt{10} to
\texttt{0}.

%%%%%%%%%%%%%%%%%%%%%%%%%%%%%%%%%%%%%%%%%%%%%%%%%%%%%%%%%%%%%%%%%%%%%%

\section{The Compiler}

The basic task of our compiler is to translate (source) programs
written in Mini into equivalent (target) programs
written in assembly language for a certain platform.
%
Even for a simple language like Mini, the translation process can
be rather complex.
%
It is therefore divided into simpler subtasks that each perform
part of the translation.
%
Figure~\ref{flow} illustrates this ``division of labor'' in more
detail.
%
Software systems of this form are often called ``pipeline''
architectures.

\begin{figure}
\centering
\includegraphics[height=0.3\textheight]{flow}
\caption{Conceptual architecture in terms of data flows
and compilation tasks.\label{flow}}
\end{figure}

There are, however, other views of the compiler as well.
%
Figure~\ref{class} illustrates the object-oriented design of
the compiler using a UML class diagram~\cite{fowler:uml}.
%
TODO

\begin{figure}
\centering
\includegraphics[width=\columnwidth]{class}
\caption{Conceptual architecture in terms of classes
and associations.\label{class}}
\end{figure}

%%%%%%%%%%

\subsection{Lexical Analysis}

Scanner
TODO

%%%%%%%%%%

\subsection{Syntactic Analysis}

Parser
TODO

%%%%%%%%%%

\subsection{Code Generation}

TODO

%%%%%%%%%%%%%%%%%%%%%%%%%%%%%%%%%%%%%%%%%%%%%%%%%%%%%%%%%%%%%%%%%%%%%%

\section{The Backends}

%%%%%%%%%%

\subsection{MIPS on SPIM}

This backend is also the default for the Mini compiler.
%
The current MIPS backend spits out code suitable for
SPIM, a popular MIPS simulator \cite{larus:spim}.
%
Compile an example like this:
%
\begin{verbatim}
  ./mini examples/prime
\end{verbatim}
%
Run the generated code as follows:
%
\begin{verbatim}
  spim examples/prime.s
\end{verbatim}
%
That's all there is to it, thanks to a very nicely
designed simulator.

%%%%%%%%%%

\subsection{MMIX}

The MMIX backend spits out code suitable for the Unix
version of Knuth's MMIXware \cite{knuth:mmixware}.
%
The process of running generated code is a little more
involved.
%
Compile an example like this:
%
\begin{verbatim}
  ./mini -x examples/prime
\end{verbatim}
%
Now use the MMIX assembler to generate an object file:
%
\begin{verbatim}
  mmixal examples/prime.mms
\end{verbatim}
%
Run the generated code as follows:
%
\begin{verbatim}
  mmix examples/prime.mmo
\end{verbatim}
%
That's all there is to it, thanks to a nicely designed
simulator.

TODO
\cite{knuth:mmix}
\cite{knuth:mmix-taocp}

%%%%%%%%%%

\subsection{LC-3}

The LC-3 backend spits out code suitable for the Unix
version of the LC-3 simulator \cite{lumetta:lc3}.
%
The process of running generated code is a little more
involved.
%
Compile an example like this:
%
\begin{verbatim}
  ./mini -l examples/4711
\end{verbatim}
%
Now use the LC-3 assembler to generate an object file:
%
\begin{verbatim}
  lc3as examples/4711.asm
\end{verbatim}
%
Also, you have to generate an object file for the support
library \texttt{newtraps.asm}:
%
\begin{verbatim}
  lc3as newtraps.asm
\end{verbatim}
%
Now you have all the pieces in place.
%
Start the LC-3 simulator:
%
\begin{verbatim}
  lc3sim
\end{verbatim}
%
At the prompt, first load the support library:
%
\begin{verbatim}
  file newtraps
\end{verbatim}
%
The load the object file that came out of the Mini
compiler:
%
\begin{verbatim}
  file examples/4711
\end{verbatim}
%
And finally run the whole thing with this command:
%
\begin{verbatim}
  continue
\end{verbatim}
%
Ah, it's magic! Or is it tragic? :-)

The current LC-3 backend doesn't properly deal with
numbers larger than 16 bit, which is why some of the
example programs don't work.
%
The \texttt{lc3as} assembler doesn't complain either,
which might be more of a problem\dots{}

%%%%%%%%%%

\subsection{Motorola 6809}

TODO
\cite{benschop:6809}
\cite{bellis:6809}

%%%%%%%%%%

\subsection{Motorola 68000}

This backend is incomplete since I cannot find a decent
M68K simulator. If you have suggestions, let me know.

TODO
find references

%%%%%%%%%%%%%%%%%%%%%%%%%%%%%%%%%%%%%%%%%%%%%%%%%%%%%%%%%%%%%%%%%%%%%%
\bibliography{mini}
\bibliographystyle{plain}
%%%%%%%%%%%%%%%%%%%%%%%%%%%%%%%%%%%%%%%%%%%%%%%%%%%%%%%%%%%%%%%%%%%%%%
\end{document}
%%%%%%%%%%%%%%%%%%%%%%%%%%%%%%%%%%%%%%%%%%%%%%%%%%%%%%%%%%%%%%%%%%%%%%

%
% Snippets of Fridtjof's original Mini.LiesMich file, for reference.
%

Dieses Verzeichnis enth�lt den Mini-Compiler, der im Amiga Programmieren
Sonderheft (Markt&Technik) beschrieben wird.

Dieser Compiler d�rfte mit weniger als 8KB Quelltext und 15KB Programmcode
einer der kleinsten Compiler �berhaupt sein. Dennoch enth�lt er alle Teile,
die auch in einem 'gro�en' Compiler zu finden sind.

Die Sprache Mini:

%%% DONE

Benutzung des Compilers:

  Der Compiler erwartet den Namen einer Mini-Quelltextdatei als Argument.
  Die Ausgabe des Compilers ist der erzeugte Assembler-Code. Er sollte in
  eine Datei umgeleitet werden. Bsp:

    Mini >Fibonacci.s Fibonacci.mini

  Nun mu� der Text noch assembliert werden (etwa mit a68k von Fish 521):

    a68k Fibonacci.s

  zuletzt wird mit

    OLink FROM Fibonacci.o TO Fibonacci

  oder mit

    BLink Fibonacci.o TO Fibonacci

  ein ausf�hrbares Programm erzeugt.


Beispielprogramme:

  Fibonacci.mini:

    Berechnet die ersten 46 Fibonacci-Zahlen

  Fak.mini:

    Berechnet die Fakult�ten 1! bis 12!

  GGT.mini:

    Berechnet den GGT zweier Zahlen (Konstanten im Programm)

  Prim.mini:

    Bestimmt die Primzahlen unter den Zahlen 1 bis 1000.


Weitere Mini-Programme:

  Wie man an Prim.mini sieht, k�nnen mit dieser einfachen Sprache auch recht
  Komplexe Dinge berechnet werden. Wer weitere Mini-Programme schreibt,
  die interessante Dinge berechnen, der schicke mit bitte seine Mini-
  Quelltexte.
