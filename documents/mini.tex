%%%%%%%%%%%%%%%%%%%%%%%%%%%%%%%%%%%%%%%%%%%%%%%%%%%%%%%%%%%%%%%%%%%%%%
%
% $Id: mini.tex 651 2005-05-11 03:36:01Z phf $
%
% Copyright (c) 2005 by Peter H. Froehlich <phf@acm.org>.
% All rights reserved.
%
% This file is part of Mini.
%
% Mini is free software; you can redistribute it and/or modify it
% under the terms of the GNU General Public License as published
% by the Free Software Foundation; either version 2 of the License,
% or (at your option) any later version.
%
% Mini is distributed in the hope that it will be useful, but
% WITHOUT ANY WARRANTY; without even the implied warranty of
% MERCHANTABILITY or FITNESS FOR A PARTICULAR PURPOSE. See the
% GNU General Public License for more details.
%
% You should have received a copy of the GNU General Public License
% along with Mini; if not, write to the Free Software Foundation,
% Inc., 51 Franklin St, Fifth Floor, Boston, MA 02110-1301, USA.
%
%%%%%%%%%%%%%%%%%%%%%%%%%%%%%%%%%%%%%%%%%%%%%%%%%%%%%%%%%%%%%%%%%%%%%%

\documentclass[letterpaper,11pt,twocolumn]{article}

\usepackage{hyperref}
\usepackage{times}
\usepackage{ftnright}
\usepackage{verbatim}
\usepackage{graphicx}

\usepackage{float}
\floatstyle{ruled}
\restylefloat{figure}

\usepackage[USletter]{vmargin}
\setmargrb{1in}{1in}{1in}{1in}

\usepackage{svn}
\SVN$Rev: 651 $
\SVN$Date: 2005-05-10 20:36:01 -0700 (Tue, 10 May 2005) $

\usepackage{color}
\definecolor{darkblue}{rgb}{0.0,0.0,0.5}
\definecolor{darkgreen}{rgb}{0.0,0.3,0.0}
\definecolor{darkred}{rgb}{0.5,0.0,0.0}
\hypersetup{
colorlinks=true,
urlcolor=darkblue,% urls
citecolor=darkred,% citation of reference
linkcolor=darkgreen,% table of contents
}

\setcounter{tocdepth}{2}

\title{
\textbf{The Programming Language Mini}%
\thanks{This document is based in part on documentation originally
written by \href{http://www.fridi.de/}{Fridtjof Siebert}. Any errors
or omissions are of course Peter's fault. :-)}
}
\author{
\href{http://www.factorial.com/forward/phf/work/}{Peter H. Fr{\"o}hlich}\\
\href{mailto:phf@acm.org}{phf@acm.org}
}
\date{\SVNDate\\(Revision \SVNRev)}

%%%%%%%%%%%%%%%%%%%%%%%%%%%%%%%%%%%%%%%%%%%%%%%%%%%%%%%%%%%%%%%%%%%%%%

\begin{document}
\maketitle
\tableofcontents

%%%%%%%%%%%%%%%%%%%%%%%%%%%%%%%%%%%%%%%%%%%%%%%%%%%%%%%%%%%%%%%%%%%%%%

\section{The Language}

Mini is a \emph{very} simple programming language.
%
In theory all computations that can be performed using any
other programming language can also be performed in Mini.
%
However, it can be rather difficult to find efficient solutions
in Mini.
%
The program \texttt{prime.mini} illustrates this fact very well.
%
If you write additional Mini programs for interesting problems,
please send them in for inclusion in future Mini distributions.

%%%%%%%%%%

\subsection{Programs}

Programs consist of the keyword \texttt{PROGRAM} followed
by (optional) variable declarations, followed by the keyword
\texttt{BEGIN}, followed by (optional) instructions, followed
by the keyword \texttt{END}:
%
\begin{quote}
\begin{verbatim}
Program =
  "PROGRAM" [VarDeclaration]
  "BEGIN" {Instruction}
  "END".
\end{verbatim}
\end{quote}

%%%%%%%%%%

\subsection{Declarations}

Variable declarations are introduced with the keyword
\texttt{VAR} followed by the list of variables separated
by commas:
%
\begin{quote}
\begin{verbatim}
VarDeclaration =
  "VAR" Identifier
  {"," Identifier}.
\end{verbatim}
\end{quote}
%
Identifiers consist of one or more letters and
are case-sensitive:
%
\begin{quote}
\begin{verbatim}
Identifier =
  Letter {Letter}.
Letter =
  "a"|"b"|...|"z"|
  "A"|"B"|...|"Z".
\end{verbatim}
\end{quote}

%%%%%%%%%%

\subsection{Instructions}

There are only three possible instructions in Mini:
%
\begin{quote}
\begin{verbatim}
Instruction =
  Assignment | While | Print.
\end{verbatim}
\end{quote}
%
Assignments store the value of an expression in a variable:
%
\begin{quote}
\begin{verbatim}
Assignment =
  Identifier "=" Expression.
\end{verbatim}
\end{quote}
%
While loops are executed as long as the value of the
controlling expression is greater than zero:
%
\begin{quote}
\begin{verbatim}
While =
  "WHILE" Expression
  "DO" {Instruction} "END".
\end{verbatim}
\end{quote}
%
Print instructions write the value of an expression to
standard output:
%
\begin{quote}
\begin{verbatim}
Print =
  "PRINT" Expression.
\end{verbatim}
\end{quote}

%%%%%%%%%%

\subsection{Expressions}

Expressions allow addition and subtraction of factors,
nothing else:
%
\begin{quote}
\begin{verbatim}
Expression =
  ["+"|"-"] Factor
  {("+"|"-") Factor}.
\end{verbatim}
\end{quote}
%
Factors are either identifiers denoting variables or
constants:
%
\begin{quote}
\begin{verbatim}
Factor =
  Identifier | Constant.
\end{verbatim}
\end{quote}
%
Constants consist of one or more digits and denote
an integer value in base 10:
%
\begin{quote}
\begin{verbatim}
Constant =
  Digit {Digit}.
Digit =
  "0"|"1"|...|"9".
\end{verbatim}
\end{quote}

%%%%%%%%%%

\subsection{Example}

Here is a simple example program showing off all of
Mini's features (except for unary negation):
%
\begin{quote}
\verbatiminput{../examples/count.mini}
\end{quote}
%
We initialize the variable \texttt{x} to the value \texttt{10};
we then execute a loop as long as \texttt{x+1} is greater than
\texttt{0}
(that is, as long as \texttt{x} is greater than \texttt{-1});
inside the loop with print the current value of \texttt{x} and
then subtract \texttt{1} from \texttt{x}
before the next iteration.
%
The resulting program ``counts down'' from \texttt{10} to
\texttt{0}.

%%%%%%%%%%%%%%%%%%%%%%%%%%%%%%%%%%%%%%%%%%%%%%%%%%%%%%%%%%%%%%%%%%%%%%

\section{The Compiler}

The basic task of our compiler is to translate (source) programs
written in Mini into equivalent (target) programs
written in assembly language for a certain platform.
%
Even for a simple language like Mini, the translation process can
be rather complex.
%
It is therefore divided into simpler subtasks that each perform
part of the translation.
%
Figure~\ref{flow} illustrates this ``division of labor'' in more
detail.
%
Software systems of this form are often called ``pipeline''
architectures.

\begin{figure}
\centering
\includegraphics[height=0.3\textheight]{flow}
\caption{Conceptual architecture in terms of data flows
and compilation tasks.\label{flow}}
\end{figure}

The \emph{scanner} performs the task of
\emph{lexical analysis}:
%
It processes the stream of characters that initially
represent the source program and identifies larger
logical units called \emph{tokens}.
%
For example, the three characters ``\texttt{137}'' are
``translated'' into the token \texttt{Constant<137>},
while the four characters ``\texttt{help}'' are
``translated'' into the token \texttt{Identifier<"help">}.
%
A \emph{token} is a string of characters that ``makes
sense'' as a whole in the programming language we are
translating.
%
Other examples for tokens are keywords like ``\texttt{END}''
or special symbols like \texttt{=} for assignment.

TODO

However, Figure~\ref{flow} is only one of many views of
a compiler.
%
It emphasizes the three main tasks of lexical analysis,
syntactic analysis, and code generation.
%
It does not, however, show how those tasks (also called
``phases'') interact.

TODO

In particular, the three ``phases'' of scanner, parser,
and code generator do not run completely independently
in our compiler.
%
Figure~\ref{class} illustrates the object-oriented design
of the compiler using a UML class diagram~\cite{fowler:uml}.
%
This view shows that the parser is ``in control'' during
the whole compilation process.
%
It uses the scanner to request tokens whenever it needs
to access the source file.
%
It also uses the code generator whenever it needs to
access the target file.

\begin{figure}
\centering
\includegraphics[width=\columnwidth]{class}
\caption{Conceptual architecture in terms of classes
and associations.\label{class}}
\end{figure}

%%%%%%%%%%

\subsection{Lexical Analysis}

Scanner
TODO

%%%%%%%%%%

\subsection{Syntactic Analysis}

Parser
TODO

%%%%%%%%%%

\subsection{Code Generation}

TODO

%%%%%%%%%%%%%%%%%%%%%%%%%%%%%%%%%%%%%%%%%%%%%%%%%%%%%%%%%%%%%%%%%%%%%%

\section{The Backends}

%%%%%%%%%%

\subsection{MIPS on SPIM}

This backend is also the default for the Mini compiler.
%
The current MIPS backend spits out code suitable for
SPIM, a popular MIPS simulator \cite{larus:spim}.
%
Compile an example like this:
%
\begin{verbatim}
  ./mini examples/prime
\end{verbatim}
%
Run the generated code as follows:
%
\begin{verbatim}
  spim examples/prime.s
\end{verbatim}
%
That's all there is to it, thanks to a very nicely
designed simulator.

%%%%%%%%%%

\subsection{MMIX}

The MMIX backend spits out code suitable for the Unix
version of Knuth's MMIXware \cite{knuth:mmixware}.
%
The process of running generated code is a little more
involved.
%
Compile an example like this:
%
\begin{verbatim}
  ./mini -x examples/prime
\end{verbatim}
%
Now use the MMIX assembler to generate an object file:
%
\begin{verbatim}
  mmixal examples/prime.mms
\end{verbatim}
%
Run the generated code as follows:
%
\begin{verbatim}
  mmix examples/prime.mmo
\end{verbatim}
%
That's all there is to it, thanks to a nicely designed
simulator.

TODO
\cite{knuth:mmix}
\cite{knuth:mmix-taocp}

%%%%%%%%%%

\subsection{LC-3}

The LC-3 backend spits out code suitable for the Unix
version of the LC-3 simulator \cite{lumetta:lc3}.
%
The process of running generated code is a little more
involved.
%
Compile an example like this:
%
\begin{verbatim}
  ./mini -l examples/4711
\end{verbatim}
%
Now use the LC-3 assembler to generate an object file:
%
\begin{verbatim}
  lc3as examples/4711.asm
\end{verbatim}
%
Also, you have to generate an object file for the support
library \texttt{newtraps.asm}:
%
\begin{verbatim}
  lc3as newtraps.asm
\end{verbatim}
%
Now you have all the pieces in place.
%
Start the LC-3 simulator:
%
\begin{verbatim}
  lc3sim
\end{verbatim}
%
At the prompt, first load the support library:
%
\begin{verbatim}
  file newtraps
\end{verbatim}
%
The load the object file that came out of the Mini
compiler:
%
\begin{verbatim}
  file examples/4711
\end{verbatim}
%
And finally run the whole thing with this command:
%
\begin{verbatim}
  continue
\end{verbatim}
%
Ah, it's magic! Or is it tragic? :-)

The current LC-3 backend doesn't properly deal with
numbers larger than 16 bit, which is why some of the
example programs don't work.
%
The \texttt{lc3as} assembler doesn't complain either,
which might be more of a problem\dots{}

\subsubsection{Extended Arithmetic}

The basic LC-3 code generator uses the processor's
native 16-bit arithmetic.
%
Support for 32-bit arithmetic must be implemented
in terms of the existing 16-bit arithmetic.
%
For this purpose, we consider the eight 16-bit
registers \texttt{R0}\dots{}\texttt{R7} as four
32-bit register \texttt{L0}\dots{}\texttt{L3}.
%
Register \texttt{R0} holds the high 16-bit of
\texttt{L0} while register \texttt{R1} holds
the low 16-bit of \texttt{L0} and so on for
the remaining registers.

We now perform a 32-bit addition by performing two
16-bit additions, one for the low word and one for
the high word.
%
However, we also have to take into account the
possibility of overflow when adding the two low
16-bit registers.
%
If there is an overflow, we have to perform yet
another addition to add 1 to the high 16-bit
register.
%
The problem with this is that the LC-3 does not
provide a carry bit that we could use for this
purpose.
%
Instead, we have to analyze the possible cases in
which an overflow arises using explicit checks.

If both low words are positive (bit 15 is 0) we can
not possibly generate an overflow; note that we talk
about an overflow \emph{into} the high word, not an
overflow \emph{inside} the low word.
%
If both low words are negative (bit 15 is 1) we will
certainly generate an overflow.
%
If the two low words have different signs, we can only
tell if there was an overflow \emph{after} we perform
the actual addition.
%
If the result is negative, there was no overflow, but
if the result is positive, there was an overflow.

TODO

%%%%%%%%%%

\subsection{Motorola 6809}

TODO
\cite{benschop:6809}
\cite{bellis:6809}

\subsubsection{Extended Arithmetic}

The basic 6809 code generator uses the processor's
native 16-bit arithmetic:
%
The two 8-bit accumulators \texttt{A} and \texttt{B}
are used as the 16-bit accumulator \texttt{D}.
%
Instructions such as \texttt{ADDD} are used to add
16-bit quantities from memory to \texttt{D}.

Support for 32-bit arithmetic could have been added
in a style similar to the LC-3 above.
%
However, the 6809 offers another alternative as well.
%
Unlike the 16-bit \texttt{ADD} instruction, the 8-bit
\texttt{ADC} instruction allows additions that take a
carry-bit into account.
%
Unlike the complicated an ugly branching logic that
was necessary for the LC-3, we can therefore perform
32-bit addition simply by doing four 8-bit additions
with carry.

TODO

%%%%%%%%%%

\subsection{Motorola 68000}

This backend is incomplete since I cannot find a decent
M68K simulator. If you have suggestions, let me know.

TODO
find references

%%%%%%%%%%

\subsection{Motorola PowerPC}

The PowerPC backend spits out code suitable for Mac OS X
10.3 (and hopefully later).
%
Yes, this is actually a native-code backend, no simulator
needed; you do need a Mac though\dots{}
%
Compile an example like this:
%
\begin{verbatim}
  ./mini -p examples/prime
\end{verbatim}
%
Now use the \texttt{as} assembler to generate an object file:
%
\begin{verbatim}
  as -o examples/prime.o
    examples/prime.s
\end{verbatim}
%
Now use the \texttt{ld} linker to link against the minimum
set of libraries:
%
\begin{verbatim}
  ld -o examples/prime
    -lcrt1.o examples/prime.o
    -lSystem
\end{verbatim}
%
Now you can finally run the whole thing with like this:
%
\begin{verbatim}
  ./examples/prime
\end{verbatim}
%
Pretty impressive, huh? :-)

TODO
find references

%%%%%%%%%%

\subsection{Intel 80386 (aka IA-32)}

The Intel 80386 backend spits out code suitable for Linux
boxes.
%
Yes, this is another native-code backend, no simulator
needed; you do need a Linux box though\dots{}
%
Compile an example like this:
%
\begin{verbatim}
  ./mini -i examples/prime
\end{verbatim}
%
Now use the \texttt{as} assembler to generate an object file:
%
\begin{verbatim}
  as -o examples/prime.o
    examples/prime.s
\end{verbatim}
%
Now use the \texttt{ld} linker to link against the minimum
set of libraries:
%
\begin{verbatim}
  ld -o examples/prime
    examples/prime.o
\end{verbatim}
%
Now you can finally run the whole thing with like this:
%
\begin{verbatim}
  ./examples/prime
\end{verbatim}
%
Pretty impressive, huh? :-)

TODO
find references

%%%%%%%%%%%%%%%%%%%%%%%%%%%%%%%%%%%%%%%%%%%%%%%%%%%%%%%%%%%%%%%%%%%%%%
\bibliography{mini}
\bibliographystyle{plain}
%%%%%%%%%%%%%%%%%%%%%%%%%%%%%%%%%%%%%%%%%%%%%%%%%%%%%%%%%%%%%%%%%%%%%%
\end{document}
%%%%%%%%%%%%%%%%%%%%%%%%%%%%%%%%%%%%%%%%%%%%%%%%%%%%%%%%%%%%%%%%%%%%%%

%
% Snippets of Fridtjof's original Mini.LiesMich file, for reference.
%

Dieses Verzeichnis enth�lt den Mini-Compiler, der im Amiga Programmieren
Sonderheft (Markt&Technik) beschrieben wird.

Dieser Compiler d�rfte mit weniger als 8KB Quelltext und 15KB Programmcode
einer der kleinsten Compiler �berhaupt sein. Dennoch enth�lt er alle Teile,
die auch in einem 'gro�en' Compiler zu finden sind.

Die Sprache Mini:

%%% DONE

Benutzung des Compilers:

  Der Compiler erwartet den Namen einer Mini-Quelltextdatei als Argument.
  Die Ausgabe des Compilers ist der erzeugte Assembler-Code. Er sollte in
  eine Datei umgeleitet werden. Bsp:

    Mini >Fibonacci.s Fibonacci.mini

  Nun mu� der Text noch assembliert werden (etwa mit a68k von Fish 521):

    a68k Fibonacci.s

  zuletzt wird mit

    OLink FROM Fibonacci.o TO Fibonacci

  oder mit

    BLink Fibonacci.o TO Fibonacci

  ein ausf�hrbares Programm erzeugt.


Beispielprogramme:

  Fibonacci.mini:

    Berechnet die ersten 46 Fibonacci-Zahlen

  Fak.mini:

    Berechnet die Fakult�ten 1! bis 12!

  GGT.mini:

    Berechnet den GGT zweier Zahlen (Konstanten im Programm)

  Prim.mini:

    Bestimmt die Primzahlen unter den Zahlen 1 bis 1000.


